\chapter{Experimento}
A primeira abordagem para realizar a montagem experimental foi construir uma placa de circuito impresso com o circuito do Mazzilli. Porém, o circuito planejado drenaria uma corente de 30A e oscilaria numa frequência em torno de 400kHz. Utilizando [] para calcular qual deveria ser o tamanho da trilha, chegamos a valores que não seriam práticos. Portanto a ideia era fazer a montagem em outro material e fazer as ligações por cabos que se usam em instalação elétrica, garantindo assim o funcionamento adequado. Para isto, foi feita a montagem experimental em cima de uma placa de madeira, que é um péssimo condutor, funcionando como um ótimo isolante. 
\section{Transistores}
Para podermos atender os requisitos citados em [] foi necessário a escolha de um IRFP250N, onde podemos conferir suas especificações [] na \autoref{tabela-car-irfp}.


\begin{table}[htb]
\IBGEtab{%
  \caption{Características do IRFP250N}%
  \label{tabela-car-irfp}
}{%
  \begin{tabular}{cccc}
  \toprule
   & Parâmetro & Valor & Unidade \\
  \midrule
  $I_{DMáx}$ & Corrente contínua de Dreno & $30$ & A \\
  $P_{DMáx}$ & Dissipação de potência & $214$ & W \\
  $V_{GS}$ & Tensão de gate para source & $\pm 20$ & V \\
  \bottomrule
\end{tabular}%
}{%
  \fonte{International Rectifier}%
  \nota{Os valores máximos são para uma temperatura de $25^oC$.}%
  \nota[Anotações]{* Existem os capacitores eletrolíticos bipolares porém não possuem capacitâncias tão altas.}%
  }
\end{table}


\section{Capacitores}
Outro problema prático é que o capacitor deve ser apolar, pois o sinal excursionado nele ora é positivo, ora é negativo. Além disto, a corrente AC que passa está na ordem de dezenas de amperes, havendo a necessidade de realizar associações em paralelo para dividir a corrente entre os componentes até um valor aceitável. A \autoref{tabela-cap-comp} mostra os materiais que são comumente fabricados o capacitor. Podemos observar que o melhor material para agir em conjunto com um circuito que irá oscilar em alta frequência, com uma alta tensão e maior fator de qualidade possível, é o polipropileno. No projeto, foi utilizado 3 capacitores, modelo B32653 [].


\begin{table}[htb]
\IBGEtab{%
  \caption{Comparativo entre os tipos de capacitores.}%
  \label{tabela-cap-comp}
}{%
  \begin{tabular}{cccccc}
  \toprule
  Tipo & Polaridade & Potência & Escala & ESR & Frequência de Operação \\
  \midrule
  Cerâmica & Não & Baixa & $\mu F$ & Baixo & Alta \\
  Poliestireno & Não & Alta & $pF$ & Baixo & Muito Alta \\
  Poliester & Não	& Alta & $\mu F$ & Baixo & Muito Alta \\
  Polipropilêno & Não & Alta & $\mu F$ & Muito baixo & Muito Alta \\
  Eletrolítico & Não* & Alta & $F$ & Alta & Baixa \\
  \bottomrule
\end{tabular}%
}{%
  \fonte{Wikipédia}%
  \nota{\hyphenation{Equivalent Series Resistence }: uma medida de não idealidade do componente que mede o valor da resistência em série com o mesmo para uma determinada frequência.}%
  \nota[Anotações]{* Existem os capacitores eletrolíticos bipolares porém não possuem capacitâncias tão altas.}%
  }
\end{table}



\section{Bobina de Trabalho}

\section{Choke}
Utilizando a fórmula aproximada de Wheeler [], adaptado em [], para ficar em metros e Henrys, temos que a fórmula explicita para o cálculo da indutância é:
\begin{equation}
L = \mu_0 \frac{\pi r^2 n^2}{h + 0.9r}
\end{equation}
Onde $ L$ é a indutância desejada, $\mu_0$ é a permeabilidade no ar, $r$ o raio da seção de reta do indutor, $n$ o número de espiras e $h$ a altura da bobina. Variando os parâmetros $r$, $h$ e $L$, considerando $h = 1.5r$, temos a Figura \ref{fig_comp-indutor}.

\begin{figure}[htb]
\caption{\label{fig_comp-indutor}Número de espiras variando os parâmetros para um indutor de núcleo de ar}
\begin{center}
\includegraphics[scale=0.5]{images/comp-indutor.jpg}
\end{center}
\legend{O raio está variando de 1 a 5cm, enquanto a indutância varia de 50 a 200 $\mu H$ }
\end{figure}

Podemos observar que para uma bobina pequena temos que aumentar exponencialmente o número de espiras, além de tomar todo o cuidado para termos um espaçamento e seção de reta uniforme em cada espira. Para sanar o problema, recorremos ao uso de choques com núcleo de ferrite, comumente encontrado em fontes chaveadas. O ferrite possuí como característica alta permeabilidade magnética, baixa condutividade elétrica (prevenindo correntes parasitas) e baixa perda em alta frequência, garantido assim um menor $n$ e consequentemente um menor tamanho de choque. Encontramos a bobina em uma fonte ATX, muito utilizada em computadores (ver \autoref{fig_fonte-atx}).

\begin{figure}[htb]
\caption{\label{fig_fonte-atx}Demonstração do choke na fonte ATX}
\begin{center}
\includegraphics[scale=0.5]{images/fonte-atx.png}
\end{center}
\legend{Observe o choke, marcado com a letra 'D', de uma fonte ATX com núcleo de ferrite enrolado com várias espiras. Fonte: Wikipedia.}
\end{figure}


\section{Diodo de recuperação rápida}
O diodo que realiza o chaveamento do transistor no momento certo, ajudando a descarregar a base, deve poder responder na mesma velocidade de chaveamento necessária no circuito, com isso utilizamos o diodo BYV26A, que tem como principais aplicações, inversores em alta frequência e fontes chaveadas.
\section{Fonte}
Um dos fatores cruciais do projeto é a fonte de alimentação. Ela deve ser capaz de fornecer uma energia de 200W, uma corrente de 20A e ideal seria uma tensão regulada. No entanto, como o projeto em si é uma prova de conceito, não foi possível conseguir uma fonte de tensão regulável. Por fim, utilizamos uma bateria de chumbo, modelo FP12180, de 18Ah, mostrada na \autoref{fig_bateria}.

\begin{figure}[h]
\caption{\label{fig_bateria}Montagem experimental}
\begin{center}
\includegraphics[scale=0.1]{images/bateria.jpg}
\end{center}
\legend{Montagem experimental do circuito na base de madeira.}
\end{figure}

\section{Montagem}
Utilizamos solda de estanho e fios de $1\mbox{mm}^2$ e $1.5\mbox{mm}^2$ para a parte de baixa e alta tensão, respectivamente. A montagem final pode ser conferida na \autoref{fig_montagem}.

\begin{figure}
\caption{\label{fig_montagem}Montagem experimental}
\begin{center}
\includegraphics[scale=0.1]{images/montagem.jpg}
\end{center}
\legend{Montagem experimental do circuito na base de madeira.}
\end{figure}