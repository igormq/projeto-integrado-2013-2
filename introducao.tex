\chapter*[Introdução]{Introdução}
\addcontentsline{toc}{chapter}{Introdução}
O chuveiro elétrico é adotado por 73.1\% dos brasileiros, segundo dados da PROCEL\footnote{Programa Nacional de Conservação de Energia Elétrica}\cite{procel2007avaliaccao}, principalmente pelo seu baixo investimento inicial. O alto custo em obras para instalação de gás, solar e outros, faz com que o chuveiro elétrico seja o preferido da população brasileira. No entanto, o chuveiro elétrico convencional carece de uma boa manutenibilidade, devido à necessidade constante de troca da resistência; possui baixa relação temperatura-vazão quando comparado a soluções como aquecimento a gás; além do risco de choques. Por isso, tivemos a ideia de aplicar a tecnologia de aquecimento por indução na criação de um chuveiro. O chuveiro de aquecimento por indução tem como intuito manter a grande vantagem do chuveiro elétrico, mas prover um banho com a qualidade de técnicas que requerem uma instalação mais custosa.

Este trabalho contém uma breve introdução sobre os princípios do aquecimento por indução, diferentes topologias de osciladores usados para gerar a corrente alternada usada no aquecedor, os componentes usados na construção de um protótipo, além de sua montagem, e, por fim, serão discutidos os resultados do projeto.