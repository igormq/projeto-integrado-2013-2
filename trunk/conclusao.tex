\chapter{Resultados}
Após realizarmos testes no circuito montado, verificamos que o circuito de fato aquecia objetos metálicos. No entanto, devido à operação em potência relativamente baixa, apenas materiais ferromagnéticos tiveram aquecimento expressivo, o que nos leva a concluir que o efeito de histerese é o principal fator para o aquecimento. As correntes de Foucault não dissipam calor significativo mesmo após exposição prolongada ao campo eletromagnético. Foram feitos testes com ferro, alumínio e cobre e apenas materiais de ferro foram aquecidos.

Devido a limitações de tensão da fonte, não foi possível a operação em potências maiores que $216$W, já que com tensão de apenas $12$V, a corrente ficaria demasiadamente elevada. Testes com correntes mais altas levaram a danos nos transistores, que tiveram que ser substituídos.

Apesar do limite de potência, pudemos validar o experimento como prova de conceito, e imaginamos que com potências da ordem de grandeza de um chuveiro elétrico convencional, um chuveiro por aquecimento indutivo seja realizável.